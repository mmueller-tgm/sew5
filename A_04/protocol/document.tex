\documentclass[letterpaper, 12pt]{article}

\gdef\mytitle{SEW}
\gdef\mythema{.}

\gdef\mysubject{Fraction module documentation}
\gdef\mycourse{4CHIT 2016/17, SEW}
\gdef\myauthor{Pierre Rieger}

\gdef\myversion{1}
\gdef\mybegin{Started \today}
\gdef\myfinish{ }

\gdef\mygrade{Note:}
\gdef\myteacher{Teacher: W.Rafeiner}


\usepackage{makeidx}
\usepackage[usenames,dvipsnames,svgnames,table]{xcolor}

%!TEX root=../document.tex

\usepackage[in]{fullpage}
% Fontencoding for possible copy&paste out of PDF
\usepackage[T1]{fontenc}
\usepackage[utf8]{inputenc}
\usepackage{graphicx} 
\usepackage{wasysym}
\usepackage{textcomp}
\usepackage{sectsty}
\usepackage{caption}
\usepackage{listings}
\usepackage{array}
\usepackage{colortbl}
\usepackage{footmisc}
\usepackage{fancyhdr}
\usepackage{suffix}
\usepackage{multirow}
\usepackage{tabularx}
\usepackage{hyperref}
\usepackage{listings}
\usepackage{accsupp}
\usepackage{color}
\usepackage{url}
\usepackage[dvipsnames]{xcolor}
\usepackage[longnamesfirst,nonamebreak]{natbib}
\usepackage[headsep=1cm,headheight=3cm,hmargin=2cm,vmargin=2.5cm]{geometry}
\usepackage[nolist]{acronym}

% Definitions for Textcolor
\usepackage{color}
\definecolor{listings}{rgb}{0.96, 0.96, 0.96}
\definecolor{update}{rgb}{1, 0.8, 0.8}
\definecolor{config}{rgb}{0.8, 1, 0.8}
\definecolor{gray}{rgb}{0.4,0.4,0.4}
\definecolor{darkblue}{rgb}{0.0,0.0,0.6}
\definecolor{cyan}{rgb}{0.0,0.6,0.6}

% Java Syntaxhighligthning
% strings
\definecolor{javared}{rgb}{0.6,0,0}
% comments
\definecolor{javagreen}{rgb}{0.25,0.5,0.35}
% keywords
\definecolor{javapurple}{rgb}{0.5,0,0.35}
% javadoc
\definecolor{javadocblue}{rgb}{0.25,0.35,0.75}

\lstset{
	basicstyle=\ttfamily\small,
	keywordstyle=\bfseries\color[rgb]{0.496,0.000,0.332},
	commentstyle=\color[rgb]{0.246,0.496,0.371},
	stringstyle=\color[rgb]{0.164,0.000,0.996},
	tabsize=4,
	breaklines=true,
	numbers=left,
	numberstyle=\tiny\color{black},
	stepnumber=2,
	numbersep=8pt,
	numberstyle=\tiny,
	captionpos=b,
	xleftmargin=1cm,
	showspaces=false,
	showstringspaces=false,
	basewidth={0.53em,0.45em},
	frame=single,
	xleftmargin=1cm,
	basicstyle=\scriptsize,
}


\lstdefinestyle{Java}{
	language=Java,
	keywordstyle=\color{javapurple}\bfseries,
	stringstyle=\color{javared},
	commentstyle=\color{javagreen},
	morecomment=[s][\color{javadocblue}]{/**}{*/},
}

\lstdefinestyle{XML}{
	language=XML,
	basicstyle=\ttfamily\small,
	columns=fullflexible,
	commentstyle=\color{gray}\upshape
}

\newcommand{\noncopynumber}[1]{
	\BeginAccSupp{method=escape,ActualText={}}
	#1
	\EndAccSupp{}
}
\lstdefinestyle{bash}{
	language=bash,
	basicstyle=\ttfamily\small,
	literate={-}{{-}}{1},
	% http://tex.stackexchange.com/questions/145416/how-to-have-straight-single-quotes-in-lstlistings
	upquote=true;
	showstringspaces=false,
	%numbers=none,
	% http://tex.stackexchange.com/questions/122256/only-select-code-without-line-numbers
	numberstyle=\tiny\noncopynumber,
	breaklines=false,
	columns=fullflexible,
	basicstyle=\scriptsize
}

% http://tex.stackexchange.com/questions/83085/how-to-improve-listings-display-of-json-files
\colorlet{punct}{red!60!black}
\definecolor{background}{HTML}{EEEEEE}
\definecolor{delim}{RGB}{20,105,176}
\colorlet{numb}{magenta!60!black}



%%%%%%%%%%%%%%%%%%%%%%%%%%%%%%%%%%%%

\lhead{\mysubject}
\chead{}
\rhead{\bfseries\mythema}
\lfoot{\mycourse}
\cfoot{\thepage}
\rfoot{\myauthor}

\begin{document}
\pagenumbering{Roman}
\begin{titlepage}
	
	\begin{figure}[!h]
		\begin{flushright}
			\includegraphics[width=0.3\linewidth]{img/jdIT_tgm.png}
		\end{flushright}
	\end{figure}
	
	\vspace{2.5cm} 
	
	{\begin{center} \bfseries\huge
			\rule{17.5cm}{0.1mm}  
			\\[5mm]
			\mytitle\\[5mm]
			\mythema\\
			\rule{17.5cm}{0.1mm}  
		\end{center}}
		
		{\begin{flushright} \bfseries\Large
				\vspace{2cm}
				\mysubject\\
				\mycourse\\[10mm]
				\myauthor\\[10mm]
			\end{flushright}}
			
			{\begin{table}[!h] \bfseries\normalsize
					\begin{tabularx}{\textwidth}{lXr @{\hspace{0mm}}}
						&& Version \myversion\\
						\mygrade && \mybegin\\
						\myteacher && \myfinish\\
					\end{tabularx}
				\end{table}}
				
\end{titlepage}
			
	\parindent 0pt
	\parskip 6pt
	\newpage
	\tableofcontents
	\newpage
	\pagenumbering{arabic}
	\pagestyle{fancy}
	


\section{Introduction}
This protocol describes how the assignment was solved. \\
It discuess the use of Sphinx to build a well formated documentation for modern python code \\
as well as a guide to \textbf{TDD}(Test Driven Development)

\section{Code}
The code used to solve the assignment can be found under: \\
https://github.com/Pierre-Rieger/sew/tree/master/python. \\

\subsection{About the code}
The code for Bruch(which is refereed to as "fraction" in the following paragraphs) heavily deals with operator overloading to the point that 
almost every magic method that python provides is overridden. This is to ensure a natural work-flow that you would expect from a mathematical library.
It uses functions such as finding the least common denominator (ger: kgv), it also provides a way to be treated as a sequence by implementing \verb|__iter__|.
Herby following code is possible \verb|x, y = Bruch(1, 2)|.
The only thing the author of this document frowns upon is that the object is mutable by exposing \verb|self.zaheler and self.nenner|.\\

All overloaded operators return a new object.

\section{Tests}
As of \today all 64 of the 64 tests are passing.
To check this run \verb|python3 testfall.py| in a shell. \\
Results are following (Look at the HTML page to have a better look at the passed tests which is on github)\\
\lstinputlisting[language=XML]{tres.xml}

\section{Documentation}
Like the tests, the documentation can also be found on github!



\clearpage

\bibliographystyle{unsrt}
\bibliography{references}
\listoftables
\lstlistoflistings
\listoffigures


	
\end{document}