%!TEX root=../document.tex
\section{Aufgabenstellung}
Implementiere die folgende Aufgabenstellung in einer Programmiersprache deiner Wahl!
\subsection{Funktionalität (20 P.)}
Verwende das Decorator Pattern, um die Socket-Kommunikation zwischen einem einfachen Server und einem simplen Client (Konsole genügt) zu dekorieren! Die Plaintext-Kommunikation kann z.B. mit einer RSA- oder AES-Verschlüsselung, einer BASE64-Codierung, einem Hashwert / Fingerprint dekoriert werden. Eine implementierte Verschlüsselungsart reicht.

\subsection{Dokumentation (20 P.)}
Dokumentiere den Code ausführlich (Sphinx / JavaDoc ist nicht erforderlich).

\subsection{Protokoll (30 P.)}
Schreibe ein sauberes (Kopf- und Fußzeile, ...) Protokoll, welches Folgendes beinhaltet:

\begin{outline}
\1 UML-Klassendiagramm der verwendeten Architektur inkl. Beschreibung
    \2 Kurze allgemeine Ausarbeitung zu Design Patterns
    \2 Wie können Design Patterns unterteilt werden
    \2 Wozu Design Patterns
    \2 Übersicht existierender Design Patterns
\1 Ausarbeitung zum Decorator Pattern
    \2 Allgemeines Klassendiagramm
    \2 Grundzüge des Design Patterns (wichtige Operationen etc.) am Beispiel des implementierten Programms inkl. spezielles Klassendiagramm
    \2 Vor- und Nachteile
    \2 (Weitere) Anwendungsfälle
\1 Ausarbeitung zu einem der folgenden Design Patterns: Observer, Abstract Factory, Strategy
    \2 Allgemeines Klassendiagramm
    \2 Grundzüge des Design Patterns (wichtige Operationen etc.) mit einem kurzen eigenen Beispiel inkl. spezielles Klassendiagramm
    \2 Vor- und Nachteile
    \2 (Weitere) Anwendungsfälle
\end{outline}

\subsection{Erweiterung (30 P.)}
Implementiere eine zweite Verschlüsselungsart und zeige im Code, wie sie miteinander kombiniert werden können, um Nachrichten zu ver- und entschlüsseln!
